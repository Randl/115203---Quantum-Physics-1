\section{General properties in one dimension}
\paragraph{In one dimension, bound states are non degenerated}
$$H = \frac{P^2}{2m} + V(x)$$
Suppose $\psi_1$ and $\psi_2$ are two eigenvectors of eigenvalue $E$. Let's show that $\psi_1$ and $\psi_2$ are same eigenvector up to normalization. We know that
$$\qty(\frac{P^2}{2m} + V(x)) \ket{\psi_i} = E \ket{\psi_i}$$
Choose $i=1$, and multiply by $\psi_2$
$$\psi_2\qty(-\frac{\hbar}{2m}\pdv[2]{x}\psi_1) + V\psi_2\psi_1 = E\psi_2\psi_1$$
and similarly
$$\psi_1\qty(-\frac{\hbar}{2m}\pdv[2]{x}\psi_2) + V\psi_2\psi_1 = E\psi_2\psi_1$$
Subtracting 
$$\psi_2 \pdv[2]{x} \psi_1 - \psi_1 \pdv[2]{x} \psi_2 = 0$$
$$\qty(\psi_2 \pdv[2]{x} \psi_1 + \pdv{x}\psi_1\pdv{x}\psi_2) - \qty(\psi_1 \pdv[2]{x} \psi_2 + \pdv{x}\psi_1\pdv{x}\psi_2)= 0$$
$$\pdv{x} \qty(\psi_2\pdv{x}\psi_1 - \psi_1\pdv{x}\psi_2) = 0$$
And thus
$$\psi_2\pdv{x}\psi_1 - \psi_1\pdv{x}\psi_2 = C$$
Since states are bounded, $\psi_i \stackrel{x\to \pm \infty}{\longrightarrow} 0$, for $x\to \pm \infty$, $\psi_2\pdv{x}\psi_1 - \psi_1\pdv{x}\psi_2 =0$, and thus $C=0$.
$$\psi_2\pdv{x}\psi_1 - \psi_1\pdv{x}\psi_2 = 0$$
$$\psi_2\pdv{x}\psi_1 = \psi_1\pdv{x}\psi_2 $$
$$\ln \psi_1 = \ln \psi_2 + \tilde{C} $$
$$\psi_1 =  e^{\tilde{C}} \psi_2 $$
i.e., there is no degeneracy. This is right only for one dimension and specific Hamiltonian.
\subsection{Ehrenfest theorem and classical limit}
$$\begin{cases}
\dv{t} \expval{X} = \expval{\pdv{H}{P}}\\
\dv{t} \expval{P} = -\expval{\pdv{H}{X}}\\
\end{cases}$$
\paragraph{Ehrenfest theorem}
For time-independent $\Omega$ 
$$\dv{t} \mel{\phi}{\Omega}{\psi} = -\frac{i}{\hbar} \mel{\phi}{\comm{\Omega}{H}}{\psi}$$
\subparagraph{Proof}
$$\dv{t} \mel{\phi}{\Omega}{\psi}  = \qty(\dv{t} \bra{\phi}) \Omega \ket{\psi} + \mel{\phi}{\dv{t}\Omega}{\psi} +  \bra{\phi} \Omega \qty(\dv{t}\ket{\psi})$$
From Shr\"{o}dinger equation

$$\dv{t} \mel{\phi}{\Omega}{\psi}  = \frac{i}{\hbar} \bra{\phi} H \Omega \ket{\psi} + \mel{\phi}{\dv{t}\Omega}{\psi} +  \frac{i}{\hbar}\bra{\phi} \Omega H\ket{\psi} =-\frac{i}{\hbar} \mel{\phi}{\comm{\Omega}{H}}{\psi} + \mel{\phi}{\dv{t}\Omega}{\psi} $$
\paragraph{}
For Hamiltonian of form
$$H = \frac{P^2}{2m} + V(x)$$
$$\dv{t}\expval{X} = -\frac{i}{\hbar} \expval{\comm{X}{H}}$$
Since each function of operator commutes with operator
$$\dv{t}\expval{X} = -\frac{i}{\hbar} \expval{\comm{X}{\frac{P^2}{2m}}}$$
$$\comm{X}{P^2} =P\comm{X}{P} + \comm{X}{P}P = 2i\hbar P$$
$$\dv{t} \expval{X} = -\frac{i}{\hbar} \frac{1}{2m} 2i\hbar \expval{P} $$
$$\dv{t} \expval{X} =  \frac{\expval{P}}{m}   $$

For more general case
$$H(X,P) = \sum_{n,m} a_{nm} X^nP^m$$
We want to calculate $\comm{X}{H}$
$$\comm{X}{H} = \comm{X}{\sum_{n,m} a_{nm} X^nP^m} = \sum_{n,m} a_{nm}\comm{X}{ X^nP^m} = \sum_{n,m} a_{nm}\qty(X^n\comm{X}{ P^m}+\comm{X}{X^n }P^m)  = \sum_{n,m} a_{nm}X^n\comm{X}{ P^m}$$
We can show by induction that $\comm{X}{ P^m} = mi\hbar P^{m-1}$ and then
$$\comm{X}{H} =  i\hbar  \expval{\pdv{H}{P}}$$
and

$$\dv{t} \expval{X} =  \expval{\pdv{H}{P}} $$

For $P$ we get $\dv{t} \expval{P} = -i\hbar \expval{\comm{P}{H}}$.
$$\comm{P}{H} = \comm{P}{\sum_{n,m} a_{nm} X^nP^m} = \sum_{n,m} a_{nm}\comm{P}{ X^n}P^m$$
$$\comm{P}{X} = -i\hbar$$
By induction we can show $\comm{P}{X^n} = -ni\hbar X^{n-1}$.
$$\comm{P}{H} = \sum_{n,m} -a_{nm}ni\hbar X^{n-1}P^m = -i\hbar \pdv{H}{X}$$
\paragraph{} Take a look at $H = \frac{P^2}{2m} + V(X)$ such that $\pdv{H}{X} = V'(X)$. Define $x_0 = \expval{X}$ such that
$$V'(X) = V'(x_0) I + (X-Ix_0)V''(x_0) + \frac{1}{2}(X-Ix_0)^2V'''(x_0) + \dots$$
We want to compare $V'(\expval{X})$ and $\expval{V'(X)}$.
$$\expval{V'(X)} = V'(x_0) + (x_0-x_0)V''(x_0) + \frac{1}{2} \Delta x^2 V'''(x_0)+\dots = V'(x_0) +\frac{1}{2} \Delta x^2 V'''(x_0)+\dots$$

We've got that classical equations can be applied if $\Delta x^2$ is small.
\paragraph{Uncertainty principle}
We'll show that for $\Omega$ and $\Lambda$ such that $\comm{\Omega}{\Lambda} = i \uparrow$ 
$$\Delta \Omega^2 \Delta \Lambda^2 = \frac{1}{4} \expval{\uparrow}^2$$