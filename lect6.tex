\subsection{Fourier transform operator}
Define space with basis $\ket{x}$ such that $\braket{x}{f} = f(x)$ and $\braket{x}{x'} = \delta(x-x')$
Define identity operator
$$I = \int \dyad{x}{x} \dd{x}$$
Define also operator $K$ such that
$$\mel{x}{K}{f} = - i\pdv{f}{x}$$
$K$ is Hermitian if $\eval{f \vdot g}_a^b = 0$ for any $f$,$g$.

It's eigenvalues $k$ are such that
$$\braket{x}{k} = \frac{1}{\sqrt{2\pi}} e^{ikx}$$
Also for $f$
$$\ket{f} = \int f(x) \ket{x} \dd{x} = \int \hat{f} (k) \ket{k} \dd{k} $$
when
$$\braket{k}{f} = \hat{f}(k)$$

Define operator $X$ such that $X\ket{x} = x\ket{x}$. Obviously, $X$ is Hermitian.
$$\mel{x}{X}{f} = \qty(\mel{f}{X^\dagger}{x})^* =  \qty(\mel{f}{X}{x})^* = \qty(x\braket{f}{x})^* = x^* \braket{x}{f} = xf(x) $$

Lets calculate $\comm{X}{K}$:
$$\mel{x}{XK}{f} = \mel{x}{XIK}{f} = \int \mel{x}{X}{x'} \mel{x'}{K}{f} \dd{x'} = \int x'\delta(x-x') \qty(-i\pdv{f}{x'}) \dd{x'} = -ix \pdv{f}{x}$$
Now
$$\mel{x}{KX}{f} = \int \mel{x}{K}{x'} \mel{x'}{X}{f} \dd{x'} = \int -i \delta(x-x') \pdv{x'} \qty(x'f(x)) \dd{x'} = -i \pdv{x}\qty(xf(x)) = -if(x) -ix\pdv{f}{x}  $$
Thus
$$\mel{x}{\comm{X}{K}}{f} = -ix\pdv{f}{x} - \qty(-if(x) - ix\pdv{f}{x}) = if(x) = i\mel{x}{I}{f}$$
i.e.
$$\comm{X}{K} = iI$$

We suppose that our space is complete (i.e. each Cauchy sequence converges). Complete inner product space is called Hilbert space.

\section{Quantum mechanics principles}
\begin{enumerate}
	\item Physical state is described by vector $\ket{\psi}$ in Hilbert space
	\item Given state $\ket{\psi}$, probability to measure it in state $\varphi$ is
	$$P = \frac{\abs{\braket{\varphi}{\psi}}^2}{\braket{\varphi}{\varphi} \braket{\psi}{\psi}}$$
	\item Measurable quantities are described by Hermitian operator $\Omega$:
	\begin{enumerate}
		\item Result of measurement is one of eigenvalues of $\Omega$
		\item A state corresponding to measurement of value $\omega$ is $\ket{\omega}$.
		\item After measurements, the state will be corresponding eigenvector
	\end{enumerate} 
	\item Given state $\psi$ at $t=0$, a state at time $t$ is given by following PDE:
	$$i\hbar \pdv{t} \ket{\psi}  = H\ket{\psi}$$ 
	which is called Schr\"{o}dinger equation
\end{enumerate}

\paragraph{Position and momentum operators}
Position operator is $X$ and momentum operator $P = \hbar K$.
\paragraph{Consequences of principles}
\begin{enumerate}
	\item Particle on line is described with vector in infinite-dimensional space.
	\item If $\ket{\psi}$ and $\ket{\varphi}$ are physical states, $\ket{\varphi} + \ket{\psi}$ is physical state too.
	\item It's impossible to predict results of measurements, only probability of results.
	\item Norm of the vector is not important, since we always can normalize them.
	\item Results of measurements of $\Omega$ are only $\omega_i$.
	\item If $\ket{\psi} = \ket{\omega}$, such that $\Omega \ket{\omega} = \omega\ket{\omega}$ then measurement of $\Omega$ will always give $\omega$.
	\item Classical physical quantity which depends on $x$ and $p$, its quantum analogue is same function of operators $X$ and $P$: 
	$$\Omega(x,p) \Rightarrow \Omega(X,P)$$
	The order is important, since $\comm{X}{P} \neq 0$. Some operators are defined only in quantum case, e.g. $\Omega = \comm{X}{P}$. Thus the right way to proceed is to define quantum quantities and then to check what happens in classical limit.
	\item If there is degeneracy, i.e., some eigenvalue of the operator has multiple eigenvectors, the probability is a sum of probabilities:
	$$P = \abs{\braket{\omega,1}{\psi}}^2 + \abs{\braket{\omega,2}{\psi}}^2$$
\end{enumerate}
\paragraph{Expectation and variance of measurement}
Expectation of measurement can be calculated as
$$\mel{\psi}{\Omega}{\varphi} = \sum_i \omega_i P_\psi(\omega)$$
and variance
$$(\Delta \Omega)^2 = \mel{\psi}{\qty(\Omega - \mel{\psi}{\Omega}{\psi})^2}{\psi} =  \mel{\psi}{\Omega^2 - \qty(\mel{\psi}{\Omega}{\psi})^2}{\psi}$$
\subsection{Schr\"{o}dinger equation}
\paragraph{Examples of Hamiltonian}
For free particle
$$H = \frac{P^2}{2m}$$
For harmonic oscillator
$$H = \frac{P^2}{2m} + \frac{1}{2}\omega^2 X^2$$

Suppose Hamiltonian is independent on time. We want to diagonalize Hamiltomnian.