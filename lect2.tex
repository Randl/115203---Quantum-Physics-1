For example
$$R\ket{s}= \ket{b}\braket{w}{s}-\ket{w}\braket{b}{s}=\ket{b}\frac{1}{\sqrt{2}}+ \ket{w}\frac{1}{\sqrt{2}} = \ket{h}$$

\paragraph{Examples}
\begin{itemize}
	\item $$W = \op{w}{b} + \op{w}{w}$$
	\item $$T = \op{h}{b} + \op{s}{w}$$
	\item $$Q = 2\op{w}{w} + 3\op{b}{b}$$
	This is not unitary operator, since in doesn't conserves norm.
\end{itemize}

\paragraph{Basis change}
To change from basis to basis we just substitute the values of old basis in a new one:
$$T = \ket{h} \left(\frac{1}{\sqrt{2}}\bra{h} - \frac{1}{\sqrt{2}}\bra{s}\right) + \ket{s} \left(\frac{1}{\sqrt{2}}\bra{h} + \frac{1}{\sqrt{2}}\bra{s}\right) =  \frac{1}{\sqrt{2}}\op{h}{h} - \frac{1}{\sqrt{2}}\op{h}{s} + \frac{1}{\sqrt{2}}\op{s}{h} + \frac{1}{\sqrt{2}}\op{s}{s}$$

\paragraph{Hermitian operators}
We want to build an operator $C$ such that for state $\ket{\psi}$ it will return us the average of $\psi$:
$$\braket{\psi|C}{\psi} = \langle \psi \rangle$$
If we give a value of $-1$ to white and value of $1$ to black:
$$\mel{\psi}{C}{\psi} = -1 \cdot P(\psi = w) + 1 \cdot P(\psi = b) = - \abs{\braket{w}{\psi}}^2 + \abs{\braket{b}{\psi}}^2 = -\braket{\psi}{w}\braket{w}{\psi}+\braket{\psi}{b}\braket{b}{\psi}$$
Thus
$$C = -\op{w}{w}+\op{b}{b}$$

For example, for $\psi = \frac{1}{\sqrt{3}}\ket{w} + \sqrt{\frac{2}{3}}\ket{b}$:
$$\mel{\psi}{C}{\psi} = \bra{\psi} \left( -\op{w}{w}+\op{b}{b} \right)\left( \frac{1}{\sqrt{3}}\ket{w} + \sqrt{\frac{2}{3}}\ket{b} \right) = \bra{\psi} \left(-\frac{1}{\sqrt{3}}\ket{w} + \sqrt{\frac{2}{3}}\ket{b}\right) = \frac{1}{3}$$


\paragraph{Additional example}
Suppose we have two properties - color and temperature. Possible values are red, green, blue for color and hot, lukewarm, cold for temperature.

If we measure temperature of red or green particle we get lukewarm of hot with equal probability. If we measure temperature of blue particle we get cold surely.

How can we write $\ket{b}$, $\ket{g}$, $\ket{r}$ in basis of $\ket{c}$, $\ket{l}$, $\ket{h}$?

From first experiment
$$\ket{r} =  \frac{1}{\sqrt{2}} \ket{l} + \frac{1}{\sqrt{2}} \ket{h} $$
$$\ket{g} =   \frac{1}{\sqrt{2}} \ket{l} - \frac{1}{\sqrt{2}} \ket{h} $$

Note that we need minus since else we'd get $\braket{g}{r} = 1$.
From last experiment
$$\ket{b}= \ket{c}$$

Lets denote $h=1$, $l=0$, $c=-1$ and build temperature operator. It's characterized by
$$\begin{cases}
T\ket{h} = \ket{h}\\
T\ket{l} = 0\\
T\ket{c} = -\ket{c}
\end{cases}$$
And thus an operator is
$$T = \op{h}{h} - \op{c}{c}$$

Now let's calculate temperature of state $\ket{g}$:
$$\mel{g}{T}{g} = \left(\frac{1}{\sqrt{2}} \ket{l} - \frac{1}{\sqrt{2}} \ket{h}\right)T \left(\frac{1}{\sqrt{2}} \ket{l} - \frac{1}{\sqrt{2}} \ket{h} \right) = \frac{1}{2}  \left(\ket{l} -\ket{h}\right)T \left( \ket{l} -\ket{h} \right) =-\frac{1}{2}  \left(\ket{l} -\ket{h}\right)\ket{h} = \frac{1}{2}  $$

We can also calculate variance:
$$\sigma^2 = \mel{g}{T^2}{g} - \big(\mel{g}{T}{g}\big)^2$$

\paragraph{Example}
We have a particle can be at one of five points. So possible states are $\ket{1}$,$\ket{2}$,$\ket{3}$,$\ket{4}$,$\ket{5}$.

If we want to describe a particle on finite 1D interval, we'll divide it into $N$ equal intervals of length $a$. We are interested in $a \to 0$, thus number of dimensions goes to infinity.

\paragraph{double slit experiment}